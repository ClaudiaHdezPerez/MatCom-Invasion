\documentclass{article}
\usepackage[utf8]{inputenc}
\usepackage{amsmath}
\usepackage{graphicx}
\title{Matcom Invasion}
\author{Claudia Herández Pérez \&\& Joel Aparicio Tamayo}
\date{\today}

\begin{document}

\maketitle

\begin{figure}[h!]
    \centering
    \includegraphics[width=0.5\textwidth]{OIG2.Ym9.jpeg}
    \caption{Ojalá se viera así, pero no, es en consola :(}
\end{figure}

\section{Introducción}
En este informe, se detalla la implementación de un juego de arcade clásico en lenguaje C, Space Invaders, 
con características modernas como multi-disparos, manejo de múltiples enemigos, y una experiencia de usuario interactiva.

\section{Arquitectura del Juego}
\subsection{Multi-Hilos}
El juego implementa hilos para permitir el disparo simultáneo y la gestión de varios enemigos en pantalla, usando un hilo
individual para cada bala y para cada enemigo.

\subsection{Algoritmo de Reemplazo de Páginas}
Se utiliza un algoritmo de reemplazo de páginas para determinar dinámicamente la posición de los enemigos en la pantalla, 
es decir, para tomar la decisión de por que sitio de la pantalla fueran apareciendo.

\section{Programación Orientada a Eventos}
La interacción del usuario con el juego se maneja a través de eventos, específicamente el 
reconocimiento de teclas para mover la nave y disparar. Para esto se usó la biblioteca ncurses y 
en específico la función getch();

\section{Scheduling}
Se aplica Round Robin (RR) algoritmo de scheduling para asignar y decidir la cantidad de vidas de los enemigos dado que 
los enemigos pueden tener 1, 3 o 5 vidas, y un caso especial, 15 vidas.

\section{Multimedia}
\subsection{Música}
El juego incluye una banda sonora que mejora la experiencia del usuario.

\subsection{Guardado de Records}
Los puntajes altos se guardan en un archivo \texttt{.txt} para mantener un registro de los logros del jugador.

\end{document}